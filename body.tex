\section{Introduction}

We need a workflow system and production tools which can process DESC DC2 for DP0.2. Nominally the processing starts in June 2021.
We have a milestone L3-MW-0050 in March (see \tabref{tab:miles}) for Batch system installation and configurations on IDF and L3-MW-0060 for the DP0.2 production.
The preferable way to do this would be with BPS in front of PanDA but there are potentially other solutions (see \secref{sec:potential}).

For PREOPS we should focus on PanDA in the near term get all the hooks in place and make it work for DP0.2.
Getting this in place requires some leadership and decision making. We need a product owner and manager  (see \secref{sec:team}).
This is separate from the construction side's HSC reprocessing at NCSA for development needs. The  constructions team at NCSA can continue to use Condor-based BPS for the biweekly HSC reprocessing at NCSA.

How effective these tools are will determine how effort-intensive (and successful) the large-scale processing campaigns will be.


\subsection{Milestones}
General DP0 information is  \citeds{RTN-001}. For simplicity some  milestones
are copied here in \tabref{tab:miles}. Jira is the source of truth for dates on these though some may need revising.
% This file was generated by opsMiles.py do not edit.
\tiny \begin{longtable} {|p{0.3\textwidth}  |r  |r  |r  |r |l |p{0.1\textwidth} |} \caption{FY21 Middleware Milestones \label{tab:milestones}}\\ 
\hline 
{\bf Milestone}&{\bf Jira ID}&{\bf Rubin ID}&{\bf Due Date}&{\bf Level}&{\bf Status}&{\bf Team}\\ \hline 
Read only Gen3 butler for DP0 at IDF&\jira{PREOPS-143}&L3-MW-0030&2021-03-31&3&Done&Science Users Middleware
 \\ \hline
Qserv installation on IDF&\jira{PREOPS-142}&L3-MW-0010&2021-03-31&3&Done&Science Users Middleware
 \\ \hline
PanDA based workflow system in place&\jira{PREOPS-154}&L3-MW-0050&2021-03-31&3&In Progress&Science Users Middleware
 \\ \hline
DP0.1 data loaded into Qserv on IDF&\jira{PREOPS-144}&L3-MW-0020&2021-04-30&3&Done&Science Users Middleware
 \\ \hline
Gen3 butler and pipeline task ready for DP0 production&\jira{PREOPS-156}&L3-MW-0070&2021-06-10&3&In Progress&Science Users Middleware
 \\ \hline
PanDA based workflow system with tooling (e.g. restart) added.&\jira{PREOPS-155}&L3-MW-0060&2021-06-30&3&In Progress&Science Users Middleware
 \\ \hline
Evaluate Batch Production System &\jira{PREOPS-153}&L3-MW-0040&2021-07-31&3&In Progress&Science Users Middleware
 \\ \hline
\end{longtable} \normalsize


\section {Requirements and priories}
\citeds{LDM-636} forms the formal requirements baseline.

Concisely we need the execution team to be able to run DP0.2 with minimum hand holding.
Hence the top priorities for the near term would be:
\begin{enumerate}
\item Documentation: preferably on lsst.io, enough for the execution team to kick off pipelines,  monitor and to first order troubleshoot them.
\item Workflow monitoring - some sort of web page which gives status (perhaps slightly customized)
\item Restart: Can resume an unfinished workflow. Can automatically retry jobs killed by preemption, DB connection, or other transient issues.
\item Logstash: On IDF this will be Google Logging. Any logging should end up in the same central logging system.
\item Troubleshooting failed jobs: Features to help understand non-transient failures, such as error messages aggregation and ways to reproduce failures. This kind of error usually is caused by pipeline failures and needs follow-up investigation.
\item This first version could be on Google only, though IN2P3 would be a good bonus.  It is understood this may use DOMA @ CERN but we assume BNL clear that with CERN.
\end{enumerate}
Longer term (which may not be for DP0.2)we need
\begin{itemize}
\item Installation at SLAC
\item Multi site execution with France and eventually UK as well as SLAC.
\item Campaign execution monitoring
\end{itemize}

\subsection{Timeline}
We have a milestone L3-MW-0050 in March (see \tabref{tab:miles}) for Batch system installation, documentation and configurations on IDF and L3-MW-0060 for the DP0.2 production.
We should track these two milestones: L3-MW-0050 for an initial system and L3-MW-0060 to
have the system to run DP0.2.

\subsection{Evaluation}
L3-MW-0060 will see the commencement of the processing run - we assume there may be some
hiccups  at that point. But at L3-MW-0060 + one month we should decide if this is the long term approach for Rubin Operations with DOE buy in.
Hence L3-MW-0040 is approximately the evaluation date.

Should the evaluation be positive the next phase would include setting up the back end at SLAC.

\section {Team }\label{sec:team}
SLAC obviously have long term interest in this working and on a single track so it would be good to have some SLAC oversight on the topic.
A product owner to shepherd requirements and priorities as well as  a manager to guide resources must be identified.
 Currently (all at partial fractions) the team consists of:
\begin{itemize}
\item Brian Yanny and team at FERMILAB for execution
\item Monica Adamow - Execution NCSA
\item Michelle Gower, Mikolaj Kowalik  - BPS and deployment
\item Sergey Padolski and Shuwei Ye (starting in January)- PanDA
\end{itemize}

\section{PanDA}

The PanDA (``Processing and Data Analysis'') system was created by
ATLAS at LHC to manage its massive processing efforts. In that
capacity it handles  around a million  processing jobs per day
across heterogeneous systems, supporting multiple parallel
campaigns. Its main services (PanDA, Harvester, iDDS) are driven from
a central database. The system can ingest DAGs, handle the workflow
and then the workload management. Currently PanDA cannot rerun parts
of workflow, but the feature is being actively considered for addition.

PanDA satisfies a number of criteria:
\begin{itemize}
\item Multi-site authentication
\item Multi-site processing - Harvester can be used to mitigate network traffic between sites and central workflow db; also handles site-specific submission properties allowing a range of different kinds of resources
\item Manages workflow (via iDDS) as well as workload
\item Good monitoring tools for the submitted workflow. Can be customized.
\end{itemize}

While support would be dependent on BNL expertise, several
installations of PanDA have been undertaken outside of ATLAS, so there
is experience in doing installs and of ongoing maintenance for other
organizations.

In order to demonstrate the viability and customizability of PanDA for
Rubin, BNL has set a target of doing processing with PanDA in the IDF
by the March 2021 time frame. As a part of that demonstration, they
will provide documentation of the PanDA system.

It would be additionally instructive to set up multi-site processing
to include the French Data Facility and US Data Facility during 2021.

However, campaign management is outside PanDA’s scope, so a layer on ctrl\_bps would be needed to chunk up and keep track of elements of a campaign. Ctrl\_bps would likely also need to handle resubmissions.

\subsection{PanDA backend}
PanDA began as a MySQL based system at BNL. It was switched to Oracle at ATLAS/CERN insistence when it was adopted by ATLAS and relocated to CERN in $\approx$2006. Since then PanDA has benefited from collaborative work with ATLAS Oracle experts to optimize and tune it as its usage and capabilities have grown to smoothly support DB-intensive tasks such as managing >1M concurrent jobs and fine grained processing.
PanDA’s DB interfaces are agnostic to the RDBMS back end; it is able to work with other back ends, e.g. a MySQL instance has operated in Amazon EC2 for many years.
The issues in using a non-Oracle back end
\begin{enumerate}
	\item tuning and
	\item production use of recent code/capability
\end{enumerate}
Tuning: while not an issue for relatively small scale usage, if operational scale approaches that of ATLAS (as Rubin’s will), DB tuning and optimization is important. Dedicated effort and expertise for the chosen back end will be required.
Production use of recent code/capability: PanDA is under active development, and Rubin’s use cases leverage recent/current developments, e.g. iDDS. That development takes place with Oracle as the back end. In order to promptly use new developments against another back end, an expert on that back end would need to work hand in hand with the developers to test, debug, tune and validate the new functionality.

\subsubsection{Oracle features used by PanDA}
SLAC, via Stanford, has free use of Oracle but with a given standard feature set. Oracle charges through the nose for additional features. Could you send along the list of features required for PanDA to use Oracle, please? Then we can confirm that our license conforms.


\begin{itemize}
\item Partitioning
\begin{itemize}
\item Essential for scalability, particularly as the DB grows over time. Time-wise partitioning ensures that historical data can be accessed promptly and efficiently. This is the only extended functionality used by PanDA itself.
\end{itemize}
\item Optimization tools
\begin{itemize}
\item Compression, transportable namespaces used in DB optimization, administration
\end{itemize}
\item Diagnostics, Tuning
\begin{itemize}
\item Valuable for monitoring and maintaining the database.
\end{itemize}
\item Oracle Data Guard
\begin{itemize}
\item Used for backup. Without it, some alternative robust approach to DB backup would be needed.
\end{itemize}
\end{itemize}

CERN is also concerned about minimizing Oracle license costs. One approach they are taking is to use fewer, more powerful servers since the license cost scales with core count but not core power. The ATLAS PanDA production DB will shortly be moving from a 20 core server to a server with 8 slightly more powerful cores. The present server has 512GB memory. Large memory is important to avoid disk reads as much as possible. In the new 8 core server, memory will be increased to 768GB to compensate for the reduced core count. As mentioned below, this configuration provides the system with large headroom for load fluctuations and large spikes.

The ATLAS PanDA Oracle database uses about 75TB to serve PanDA’s 15 years of ATLAS data.

\subsection{Physical resources required for PanDA at Rubin’s scale}
ATLAS PanDA resource usage:

\begin{itemize}
\item PanDA server: 9 servers x (8 cores, 16GB RAM, 500GB disk)
\item JEDI: 9 servers x (8 cores, 16GB RAM, ~500GB disk)
\begin{itemize}
\item PanDA/JEDI scale roughly with jobs per day (concurrent job count together with job duration)
\item Configuration is by design far from a performance ceiling, designed to handle large spikes (cf. below)
\end{itemize}
\item PanDA monitoring: 8 servers x (8 cores, 16GB RAM, 80GB) + 1TB shared space for logs storage.
\begin{itemize}
\item scales with operating scale (same as PanDA/JEDI servers) together with history (15 years of data, most of the monitoring usage looking at the last ~year of data)
\end{itemize}
\item Harvester: ~8 x (4 cores, 8GB RAM, 500GB disk)
\begin{itemize}
\item Harvester scales roughly with number of sites
\end{itemize}
\item Condor: 6 x (8 cores, 16GB RAM, 100GB disk)
\begin{itemize}
\item Condor scales roughly with number of job slots (which scales roughly with concurrent core count)
\end{itemize}
\item iDDS: 2 x (8 cores, 16GB RAM, 100GB disk)
\begin{itemize}
\item iDDS is relatively new, its scalability in ATLAS is being exercised in the data carousel use case, in production. We’ll learn how it scales in Rubin as we implement and exercise use cases.
\end{itemize}
\end{itemize}

We describe here the ATLAS PanDA resource usage. In estimating Rubin resource needs from this, these ATLAS PanDA metrics should be kept in mind:

\begin{itemize}
\item Steady state operation is currently ~400-450k concurrent cores on average
\item ATLAS PanDA scales smoothly to 1.4M+ concurrent cores, exercised during HPC spikes
\item Each job communicates a heartbeat to the PanDA server every 15min
\item Average job duration is 8 hours and ranges from minutes to a few days
\begin{itemize}
\item Longer (within limits) is better for efficiency, system load and limiting DB growth. Ideal duration is a couple of hours. ATLAS has a lot of short O(10min) jobs, but it's preferable to avoid them if possible.
\end{itemize}
\item About 1M jobs are processed per day on average, peaking at about 2M
\item O(1000) users active in the last year, including individual analysis
\item O(100) global processing sites
\item About 150 PanDA queues globally
\end{itemize}

\subsubsection{JEDI in ATLAS and Rubin}
While JEDI plays a role in the ATLAS Production System Prodsys2 that Rubin will not use, JEDI also provides functionality that Rubin will use. (JEDI basically operates as an integral part of the PanDA core.)  JEDI works with iDDS to provide task-based orchestration of jobs. A ForcedPhotCoaddTask in the processing pipeline may consist of thousands of jobs. Managing and monitoring them as a consolidated payload that is run with different input parameters makes use of JEDI. A hundred-job processing pipeline consists of only 14 tasks (DeblendCoaddSourcesSingleTask, MergeDetectionsTask, DetectCoaddSourcesTask, ….), which makes it significantly easier to debug problems at a particular processing step and control the execution progress.

Graph generation is done using the BPS subsystem of the Rubin middleware. Once the graph is defined, the PanDA plugin in Rubin middleware sends the definition to the iDDS server, and at that point PanDA/JEDI takes care of its execution according to the relationship between jobs/tasks and inputs specified in the Butler repository.

\subsubsection {Condor usage }
Condor has since PanDA’s inception been its most important workhorse for submitting pilot jobs and harvesting processing resources. The condor service machines handle the submission of PanDA pilots to sites (condor queues) around the world. In recent years Harvester has been developed to abstract PanDA itself away from the many kinds of execution back ends that are now supported, of which condor is one. Harvester manages the condor pilot job submission activities carried out by the condor service machines.

For ATLAS, with over 100 sites most of which use condor, the scale of the condor submission operation is large. For Rubin with far fewer sites it is a smaller task that will basically disappear inside the Harvester service serving Rubin sites.

\subsubsection{Rome  - cores and memory }
Many ATLAS PanDA components are using 8 core VMs simply because that is the configuration preferred by CERN IT. It is the largest core count configuration CERN supports. From our point of view it would be preferable to have VMs 2x or 4x the size and hence fewer of them. But 128 core/512 GB is a big step in consolidating servers that we have never tried. Horizontal scalability is efficiently supported by having several servers and adding additional as needed. Using many servers also gives -- crucially -- redundancy and resiliency. You don’t want to lose services if a huge server falls over.  Apart from these considerations, we see no reason not to try 128 core/512 GB servers. The servers don't have heavy network traffic or disk I/O requirements. In terms of power, we estimate that a 128 core/512 GB server is roughly commensurate with the power Rubin will need in the near/mid term. We assume some standard virtualization layer will be used with this server and the whole setup will be available as a number of VMs.

\subsubsection{Advanced compression}
Some tables are currently using advanced compression. The largest OLTP compressed tables however are archive tables, i.e. the rows are not updated and data is added by a job in bulk. We could probably get away with basic compression.

\subsubsection{LOBs in the db}
There are LOBs in the DB, but they are not compressed.

\subsubsection{Transportable namespaces (tablespaces)}
In ATLAS there are 2 databases for PanDA: the “active” database and the “archival" database.
\begin{itemize}
    \item The active database holds the last 3 years of data and is based on fast storage. The DBA creates yearly tablespaces.
    \item The archival database holds jobs older than 3 years. In practice the data in the archival database is never used and kept only for exceptional cases (e.g. investigation of a very old issue). Storage can be less performant, since the DB is used extremely rarely.
\end{itemize}
The tablespaces are copied from the “active” DB to the “archival” DB manually once per year.

Alternative strategies for archival could be possible.

\subsubsection{Data Guard}
From an application perspective there is no dependency at all on Data Guard or on RAC. These are setups available at CERN to provide highly available services.

Active Data Guard is used for physical standby in case the primary would fail. PanDA only relies on the primary database.

\subsubsection{Oracle server configuration}
At CERN there is a RAC cluster with three fat nodes. EACH node has 20 core+512GB mem, moving to 8core+768GB mem. Applications are clearly delimited across the nodes so that they don’t overwrite the memory of each other or affect each other. Applications could failover to other nodes in case of node unavailability, but this never happens during normal operations. The assignment during normal operation is:
\begin{itemize}
    \item One fat node for PanDA
    \item One fat node for Rucio (data management)
    \item One fat node for monitoring applications and various side services
\end{itemize}
Old CPU: 2 x Intel Xeon E5-2630 v4 (broadwell, total of 20 physical cores), 3.10 GHz max

New CPU: 2 x Intel Xeon Gold 5222 (cascade lake, total of 8 physical cores), 3.90 GHz max

PanDA developers also use a small clone of the production DB, called INTR (integration) DB, for development. I think it would be TBD by the developers whether there should be an INTR at SLAC or Rubin-focused developers use also the INTR at CERN.


\section {Potential solutions} \label{sec:potential}

Conceptually this is done in two steps: (a) workflow generation and (b) job execution.
In step (a) the workflow generation defines executable jobs and job interdependency as a graph.
In step (b) job execution includes workflow status monitoring, pausing/resuming/killing workflows, debugging/retrying failed jobs, resource usage monitoring, and relevant toolkits to facilitate execution management on a large scale.
\begin{enumerate}
\item  ctrl\_bps workflow generation + PanDA-plugin execution tools developed by BNL
\item ctrl\_bps workflow generation + Condor-plugin execution tools developed by NCSA
\item ctrl\_bps workflow generation + Pegasus as the execution tools
\item ctrl\_bps workflow generation tools can't work on IDF, one can use customized scripts to generate workflow for any execution tools.
\end{enumerate}



\section {Risks and worries}

\begin{enumerate}
\item Lack of documentation for PanDA: it is a complex system and will
  be the heart of processing.. operating for 12 years in this mode is unwise.\label{i:nodoc}
\item Dependence on an institution or individual because of
  \ref{i:nodoc}, also suggests the need to spread the expertise more
  broadly across the team.
\item Having a LOT of scripting to make a production run of any size
\item dependence on Oracle: is an open source project would you not like to depend on commercial products furthermore some of us have had bad experience with Oracle.
\end{enumerate}
