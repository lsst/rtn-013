\documentclass[DM,authoryear,toc]{lsstdoc}
% lsstdoc documentation: https://lsst-texmf.lsst.io/lsstdoc.html
\input{meta}

% Package imports go here.

% Local commands go here.

%If you want glossaries
%\input{aglossary.tex}
%\makeglossaries

\title{Near term workflow for pre-operations with PanDA}

% Optional subtitle
% \setDocSubtitle{A subtitle}

\author{%
William O'Mullane and Richard Dubois and Hsin Fang Chiang
}

\setDocRef{RTN-013}
\setDocUpstreamLocation{\url{https://github.com/lsst/rtn-013}}

\date{\vcsDate}

% Optional: name of the document's curator
% \setDocCurator{The Curator of this Document}

\setDocAbstract{%
This note outlines our current understanding of PanDA and lays out some priorities for the near term.
}

% Change history defined here.
% Order: oldest first.
% Fields: VERSION, DATE, DESCRIPTION, OWNER NAME.
% See LPM-51 for version number policy.
\setDocChangeRecord{%
  \addtohist{1}{YYYY-MM-DD}{Unreleased.}{William O'Mullane}
}


\begin{document}

% Create the title page.
%\maketitle
% Frequently for a technote we do not want a title page  uncomment this to remove the title page and changelog.
\mkshorttitle

\section{Introduction}

We need a workflow system and production tools which can process DESC DC2 for DP0.2. Nominally the processing starts in June 2021.
We have a milestone L3-MW-0050 in March (see \tabref{tab:miles}) for Batch system installation and configurations on IDF and L3-MW-0060 for the DP0.2 production.
The preferable way to do this would be with BPS in front of PanDA but there are potentially other solutions (see \secref{sec:potential}).

For PREOPS we should focus on PanDA in the near term get all the hooks in place and make it work for DP0.2.
Getting this in place requires some leadership and decision making. We need a product owner and manager  (see \secref{sec:team}).
This is separate from the construction side's HSC reprocessing at NCSA for development needs. The  constructions team at NCSA can continue to use Condor-based BPS for the biweekly HSC reprocessing at NCSA.

How effective these tools are will determine how effort-intensive (and successful) the large-scale processing campaigns will be.


\subsection{Milestones}
General DP0 information is  \citeds{RTN-001}. For simplicity some  milestones
are copied here in \tabref{tab:miles}. Jira is the source of truth for dates on these though some may need revising.
% This file was generated by opsMiles.py do not edit.
\tiny \begin{longtable} {|p{0.3\textwidth}  |r  |r  |r  |r |l |p{0.1\textwidth} |} \caption{FY21 Middleware Milestones \label{tab:milestones}}\\ 
\hline 
{\bf Milestone}&{\bf Jira ID}&{\bf Rubin ID}&{\bf Due Date}&{\bf Level}&{\bf Status}&{\bf Team}\\ \hline 
Read only Gen3 butler for DP0 at IDF&\jira{PREOPS-143}&L3-MW-0030&2021-03-31&3&Done&Science Users Middleware
 \\ \hline
Qserv installation on IDF&\jira{PREOPS-142}&L3-MW-0010&2021-03-31&3&Done&Science Users Middleware
 \\ \hline
PanDA based workflow system in place&\jira{PREOPS-154}&L3-MW-0050&2021-03-31&3&In Progress&Science Users Middleware
 \\ \hline
DP0.1 data loaded into Qserv on IDF&\jira{PREOPS-144}&L3-MW-0020&2021-04-30&3&Done&Science Users Middleware
 \\ \hline
Gen3 butler and pipeline task ready for DP0 production&\jira{PREOPS-156}&L3-MW-0070&2021-06-10&3&In Progress&Science Users Middleware
 \\ \hline
PanDA based workflow system with tooling (e.g. restart) added.&\jira{PREOPS-155}&L3-MW-0060&2021-06-30&3&In Progress&Science Users Middleware
 \\ \hline
Evaluate Batch Production System &\jira{PREOPS-153}&L3-MW-0040&2021-07-31&3&In Progress&Science Users Middleware
 \\ \hline
\end{longtable} \normalsize


\section {Requirements and priories}
\citeds{LDM-636} forms the formal requirements baseline.

Concisely we need the execution team to be able to run DP0.2 with minimum hand holding.
Hence the top priorities for the near term would be:
\begin{enumerate}
\item Documentation: preferably on lsst.io, enough for the execution team to kick off pipelines,  monitor and to first order troubleshoot them.
\item Workflow monitoring - some sort of web page which gives status (perhaps slightly customized)
\item Restart: Can resume an unfinished workflow. Can automatically retry jobs killed by preemption, DB connection, or other transient issues.
\item Logstash: On IDF this will be Google Logging. Any logging should end up in the same central logging system.
\item Troubleshooting failed jobs: Features to help understand non-transient failures, such as error messages aggregation and ways to reproduce failures. This kind of error usually is caused by pipeline failures and needs follow-up investigation.
\item This first version could be on Google only, though IN2P3 would be a good bonus.  It is understood this may use DOMA @ CERN but we assume BNL clear that with CERN.
\end{enumerate}
Longer term (which may not be for DP0.2)we need
\begin{itemize}
\item Installation at SLAC
\item Multi site execution with France and eventually UK as well as SLAC.
\item Campaign execution monitoring
\end{itemize}

\subsection{Timeline}
We have a milestone L3-MW-0050 in March (see \tabref{tab:miles}) for Batch system installation, documentation and configurations on IDF and L3-MW-0060 for the DP0.2 production.
We should track these two milestones: L3-MW-0050 for an initial system and L3-MW-0060 to
have the system to run DP0.2.

\subsection{Evaluation}
L3-MW-0060 will see the commencement of the processing run - we assume there may be some
hiccups  at that point. But at L3-MW-0060 + one month we should decide if this is the long term approach for Rubin Operations with DOE buy in.
Hence L3-MW-0040 is approximately the evaluation date.

Should the evaluation be positive the next phase would include setting up the back end at SLAC.

\section {Team }\label{sec:team}
SLAC obviously have long term interest in this working and on a single track so it would be good to have some SLAC oversight on the topic.
A product owner to shepherd requirements and priorities as well as  a manager to guide resources must be identified.
 Currently (all at partial fractions) the team consists of:
\begin{itemize}
\item Brian Yanny and team at FERMILAB for execution
\item Monica Adamow - Execution NCSA
\item Michelle Gower, Mikolaj Kowalik  - BPS and deployment
\item Sergey Padolski and Shuwei Ye (starting in January)- PanDA
\end{itemize}

\section{PanDA}

The PanDA (``Processing and Data Analysis'') system was created by
ATLAS at LHC to manage its massive processing efforts. In that
capacity it handles  around a million  processing jobs per day
across heterogeneous systems, supporting multiple parallel
campaigns. Its main services (PanDA, Harvester, iDDS) are driven from
a central database. The system can ingest DAGs, handle the workflow
and then the workload management. Currently PanDA cannot rerun parts
of workflow, but the feature is being actively considered for addition.

PanDA satisfies a number of criteria:
\begin{itemize}
\item Multi-site authentication
\item Multi-site processing - Harvester can be used to mitigate network traffic between sites and central workflow db; also handles site-specific submission properties allowing a range of different kinds of resources
\item Manages workflow (via iDDS) as well as workload
\item Good monitoring tools for the submitted workflow. Can be customized.
\end{itemize}

While support would be dependent on BNL expertise, several
installations of PanDA have been undertaken outside of ATLAS, so there
is experience in doing installs and of ongoing maintenance for other
organizations.

In order to demonstrate the viability and customizability of PanDA for
Rubin, BNL has set a target of doing processing with PanDA in the IDF
by the March 2021 time frame. As a part of that demonstration, they
will provide documentation of the PanDA system.

It would be additionally instructive to set up multi-site processing
to include the French Data Facility and US Data Facility during 2021.

However, campaign management is outside PanDA’s scope, so a layer on ctrl\_bps would be needed to chunk up and keep track of elements of a campaign. Ctrl\_bps would likely also need to handle resubmissions.

\subsection{PanDA backend}
PanDA began as a MySQL based system at BNL. It was switched to Oracle at ATLAS/CERN insistence when it was adopted by ATLAS and relocated to CERN in $\approx$2006. Since then PanDA has benefited from collaborative work with ATLAS Oracle experts to optimize and tune it as its usage and capabilities have grown to smoothly support DB-intensive tasks such as managing >1M concurrent jobs and fine grained processing.
PanDA’s DB interfaces are agnostic to the RDBMS back end; it is able to work with other back ends, e.g. a MySQL instance has operated in Amazon EC2 for many years.
The issues in using a non-Oracle back end
\begin{enumerate}
	\item tuning and
	\item production use of recent code/capability
\end{enumerate}
Tuning: while not an issue for relatively small scale usage, if operational scale approaches that of ATLAS (as Rubin’s will), DB tuning and optimization is important. Dedicated effort and expertise for the chosen back end will be required.
Production use of recent code/capability: PanDA is under active development, and Rubin’s use cases leverage recent/current developments, e.g. iDDS. That development takes place with Oracle as the back end. In order to promptly use new developments against another back end, an expert on that back end would need to work hand in hand with the developers to test, debug, tune and validate the new functionality.



\section {Potential solutions} \label{sec:potential}

Conceptually this is done in two steps: (a) workflow generation and (b) job execution.
In step (a) the workflow generation defines executable jobs and job interdependency as a graph.
In step (b) job execution includes workflow status monitoring, pausing/resuming/killing workflows, debugging/retrying failed jobs, resource usage monitoring, and relevant toolkits to facilitate execution management on a large scale.
\begin{enumerate}
\item  ctrl\_bps workflow generation + PanDA-plugin execution tools developed by BNL
\item ctrl\_bps workflow generation + Condor-plugin execution tools developed by NCSA
\item ctrl\_bps workflow generation + Pegasus as the execution tools
\item ctrl\_bps workflow generation tools can't work on IDF, one can use customized scripts to generate workflow for any execution tools.
\end{enumerate}



\section {Risks and worries}

\begin{enumerate}
\item Lack of documentation for PanDA: it is a complex system and will
  be the heart of processing.. operating for 12 years in this mode is unwise.\label{i:nodoc}
\item Dependence on an institution or individual because of
  \ref{i:nodoc}, also suggests the need to spread the expertise more
  broadly across the team.
\item Having a LOT of scripting to make a production run of any size
\item dependence on Oracle: is an open source project would you not like to depend on commercial products furthermore some of us have had bad experience with Oracle.
\end{enumerate}

\appendix

\section {Dec 2020 Planning}
This is preserved for completion as the status of the original ideas in December 2020.
\subsection {Potential solutions} \label{sec:potential}

Conceptually this is done in two steps: (a) workflow generation and (b) job execution.
In step (a) the workflow generation defines executable jobs and job interdependency as a graph.
In step (b) job execution includes workflow status monitoring, pausing/resuming/killing workflows, debugging/retrying failed jobs, resource usage monitoring, and relevant toolkits to facilitate execution management on a large scale.
\begin{enumerate}
\item  ctrl\_bps workflow generation + PanDA-plugin execution tools developed by BNL
\item ctrl\_bps workflow generation + Condor-plugin execution tools developed by NCSA
\item ctrl\_bps workflow generation + Pegasus as the execution tools
\item ctrl\_bps workflow generation tools can't work on IDF, one can use customized scripts to generate workflow for any execution tools.
\end{enumerate}



\subsection {Risks and worries}
These original risks have been accepted or are being addressed.
\begin{enumerate}
\item Lack of documentation for PanDA: it is a complex system and will
  be the heart of processing.. operating for 12 years in this mode is unwise.\label{i:nodoc}
\item Dependence on an institution or individual because of
  \ref{i:nodoc}, also suggests the need to spread the expertise more
  broadly across the team.
\item Having a LOT of scripting to make a production run of any size
\item dependence on Oracle: is an open source project would you not like to depend on commercial products furthermore some of us have had bad experience with Oracle.
\end{enumerate}

% Include all the relevant bib files.
% https://lsst-texmf.lsst.io/lsstdoc.html#bibliographies
\section{References} \label{sec:bib}
\renewcommand{\refname}{} % Suppress default Bibliography section
\bibliography{local,lsst,lsst-dm,refs_ads,refs,books}

% Make sure lsst-texmf/bin/generateAcronyms.py is in your path
\section{Acronyms} \label{sec:acronyms}
\addtocounter{table}{-1}
\begin{longtable}{p{0.145\textwidth}p{0.8\textwidth}}\hline
\textbf{Acronym} & \textbf{Description}  \\\hline

ATLAS & A Toroidal LHC Apparatus \\\hline
BNL & Brookhaven National Laboratory \\\hline
BPS & Batch Production Service \\\hline
CERN & European Organization for Nuclear Research \\\hline
CPU & Central Processing Unit \\\hline
DB & DataBase \\\hline
DBA & database administrator \\\hline
DC2 & Data Challenge 2 (DESC) \\\hline
DESC & Dark Energy Science Collaboration \\\hline
DM & Data Management \\\hline
DMTN & DM Technical Note \\\hline
DOE & Department of Energy \\\hline
DP0 & Data Preview 0 \\\hline
DR1 & Data Release 1 \\\hline
FY21 & Financial Year 21 \\\hline
GB & Gigabyte \\\hline
HPC & High Performance Computing \\\hline
HSC & Hyper Suprime-Cam \\\hline
IDF & Interim Data Facility \\\hline
IN2P3 & Institut National de Physique Nucléaire et de Physique des Particules \\\hline
IT & Information Technology \\\hline
JEDI & Job Execution and Definition Interface \\\hline
L3 & Lens 3 \\\hline
LDM & LSST Data Management (Document Handle) \\\hline
LHC & Large Hadron Collider (at CERN) \\\hline
NCSA & National Center for Supercomputing Applications \\\hline
PanDA &  Production ANd Distributed Analysis system \\\hline
RAC & Resource Allocation Committee \\\hline
RAM & Random Access Memory \\\hline
RDBMS & Relational Database Management System  \\\hline
RSP & Rubin Science Platform \\\hline
RTN & Rubin Technical Note \\\hline
SLAC & SLAC National Accelerator Laboratory \\\hline
TBD & To Be Defined (Determined) \\\hline
UKDF & United Kingdom Data Facility \\\hline
US & United States \\\hline
USDF & United States Data Facility \\\hline
VM & Virtual Machine \\\hline
bps & bit(s) per second \\\hline
\end{longtable}

% If you want glossary uncomment below -- comment out the two lines above
%\printglossaries





\end{document}
