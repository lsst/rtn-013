
\section {Dec 2020 Planning}
This is preserved for completion as the status of the original ideas in December 2020.
\subsection {Potential solutions} \label{sec:potential}

Conceptually this is done in two steps: (a) workflow generation and (b) job execution.
In step (a) the workflow generation defines executable jobs and job interdependency as a graph.
In step (b) job execution includes workflow status monitoring, pausing/resuming/killing workflows, debugging/retrying failed jobs, resource usage monitoring, and relevant toolkits to facilitate execution management on a large scale.
\begin{enumerate}
\item  ctrl\_bps workflow generation + PanDA-plugin execution tools developed by BNL
\item ctrl\_bps workflow generation + Condor-plugin execution tools developed by NCSA
\item ctrl\_bps workflow generation + Pegasus as the execution tools
\item ctrl\_bps workflow generation tools can't work on IDF, one can use customized scripts to generate workflow for any execution tools.
\end{enumerate}



\subsection {Risks and worries}
These original risks have been accepted or are being addressed.
\begin{enumerate}
\item Lack of documentation for PanDA: it is a complex system and will
  be the heart of processing.. operating for 12 years in this mode is unwise.\label{i:nodoc}
\item Dependence on an institution or individual because of
  \ref{i:nodoc}, also suggests the need to spread the expertise more
  broadly across the team.
\item Having a LOT of scripting to make a production run of any size
\item dependence on Oracle: is an open source project would you not like to depend on commercial products furthermore some of us have had bad experience with Oracle.
\end{enumerate}
